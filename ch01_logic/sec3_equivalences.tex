% -*- mode: latex; TeX-master: "ch1_logic_main.tex"; fill-column: 75 -*-
\section{Propositional Equivalences}
\subsection{Truth assignments}
For the rest of this section, we will step away from specific propositions
like ``if it is a holiday, then there is no class,'' and instead use
abstract variables, as in ``$p \limplies \lnot q$.''  We will focus on how
the truth value of compound statements like $p \limplies \lnot q$ depends
on the truth values of $p$ and $q$.

\begin{defn}
  Given a list of propositional variables, a \newterm{truth assignment} is a
  choice of ``true'' or ``false'' for each variable.
\end{defn}
In other words, a truth assignment picks out a single row in a truth table.
For example, the assignment ``$p$ is false and $q$ is true,'' picks out the
third row in Table~\ref{tab:pimpnq}.

\begin{table}[htb] \centering
  \begin{tabular}{|c|c|c|g|c|g|}
    \hline
    $p$ & $q$ & $\lnot q$ & $p \limplies \lnot q$
    & $\lnot p$ & $q \limplies \lnot p$ \\ \hline
    T & T & F & F & F & F \\ \hline
    T & F & T & T & F & T \\ \hline
    F & T & F & T & T & T \\ \hline
    F & F & T & T & T & T\\ \hline
  \end{tabular}
  \caption{Truth table for $p \limplies \lnot q$ and $q \limplies \lnot p$}
  \label{tab:pimpnq}
\end{table}

Some propositions, like $p \lor \lnot p$, are true no matter what the value
of $p$ is.  Similarly, $p \land \lnot p$ is always false.  There are
special names for these situations:

\begin{defn}
  If a compound proposition is true for all possible truth assignments, it
  is called a \newterm{tautology}.  If it is false for all truth assignments,
  it is a \newterm{contradiction}.  If it is neither a tautology nor a
  contradiction, it is a \newterm{contingency}.
\end{defn}

\subsection{Equivalence}
From Table~\ref{tab:pimpnq} we see that $p \limplies \lnot q$ and
$q \limplies \lnot p$ have the same truth table.  We would like to say that
these two propositions are ``essentially the same.''

\begin{defn}
  Two compound propositions $\alpha$ and $\beta$ are \newterm{logically
    equivalent} if they have the same value for all truth assignments.  In
  this case write $\alpha \equiv \beta$.
\end{defn}

\begin{exercise}
  Let $\alpha$ and $\beta$ be compound propositions.  Show that
  $\alpha \equiv \beta$ if and only if $\alpha \liff \beta$ is a tautology.
\end{exercise}

\begin{danger}
  The symbol $\equiv$ is \emph{not} a logical connective like
  $\land, \lor, \limplies$, and friends.  Similarly,
  $p \limplies \lnot q \equiv q \limplies \lnot p$ is not a compound
  proposition.  It is the ``meta-statement'' that $p \limplies \lnot q$ and
  $q \limplies \lnot p$ are equivalent propositions.  Some insight into why
  this must be: it would be nonsense to try to make a truth table for
  $p \limplies \lnot q \equiv q \limplies \lnot p$.
\end{danger}

\begin{example}[De Morgan's Laws] \label{exa:demorgan} We show that
  $\lnot (p \land q) \equiv \lnot p \lor \lnot q$.  We do this by checking
  that $\lnot (p \land q)$ and $\lnot p \lor \lnot q$ have the same truth
  table.  See Table~\ref{tab:demorgan}.
  \begin{table}[h]
    \centering
    \begin{tabular}{|c|c|c|g|c|c|g|}
      \hline
      $p$ & $q$ & $p \land q$ & $\lnot (p \land q)$
      & $\lnot p$ & $\lnot q$ & $\lnot p \lor \lnot q$ \\
      \hline
      T & T & T & F & F & F & F \\ \hline
      T & F & F & T & F & T & T \\ \hline
      F & T & F & T & T & F & T \\ \hline
      F & F & F & T & T & F & T \\ \hline
    \end{tabular}
    \caption{Truth table for $\lnot (p \land q)$ and
      $\lnot p \lor \lnot q$}
    \label{tab:demorgan}
  \end{table}
\end{example}

\begin{exercise} \label{exr:notnot}%
  Show that $\lnot \lnot p \equiv p$ using truth tables.
\end{exercise}

Instead of using truth tables, we may also find new equivalences by
combining known ones.

\begin{example}
  We show $\lnot (p \lor q) \equiv \lnot p \land \lnot q$:
  \begin{align*}
    \lnot (p \lor q)
    &\equiv \lnot (\lnot \lnot p \lor \lnot \lnot q)
    & &\text{by double-negation (Exercise~\ref{exr:notnot})} \\
    &\equiv \lnot \lnot (\lnot p \land \lnot q)
    & &\text{by DeMorgan's Law (Example~\ref{exa:demorgan})} \\
    &\equiv \lnot p \land \lnot q & & \text{by double-negation}
  \end{align*}
\end{example}

\subsection{Satisfiablity}
\begin{defn}
  A compound proposition $\alpha$ is called \newterm{satisfiable} if there
  exists a truth assignment for which $\alpha$ is true.  In other words,
  $\alpha$ is not a contradiction.
\end{defn}

Many concrete problems can be reduced to asking if some compound
proposition is satisfiable, and if it is, finding a satisfying assignment.

\begin{example}[Course scheduling]
  Bertram is scheduling courses for the semester.  Let $p_1$ represent the
  statement ``Bertram enrolls in Math 55 LEC 1.''  Let $p_2,p_3,p_4$
  represent similar statements for the other sections.  The statement that
  Bertram enrolls in at least one section of Math 55 is
  \[
    \phi_1 = p_1 \lor p_2 \lor p_3 \lor p_4.
  \]
  Bertram also wants to take Math 110.  Again there are four sections,
  corresponding to four variables $q_1, q_2, q_3, q_4$.  He needs to take
  at least one section:
  \[
    \phi_2 = q_1 \lor q_2 \lor q_3 \lor q_4.
  \]
  There are some scheduling conflicts: 55 LEC 3 conflicts with 110 LEC 2
  and LEC 4.  Also, 55 LEC 2 conflicts with 110 LEC 1 and LEC 3:
  \[
    \phi_3 = \lnot (p_3 \land q_2) \land \lnot (p_3 \land q_4) \land \lnot (p_2
    \land q_1) \land \lnot (p_2 \land q_3).
  \]
  Last, Bertram refuses to take any class starting at 8am, which rules out
  55 LEC 2 and 110 LEC 1:
  \[
    \phi_4 = \lnot p_2 \land \lnot q_1.
  \]
  Now, Bertram wants to know if it's possible for him to meet all these
  requirements.  This is the same as asking if the compound proposition
  $\phi_1 \land \phi_2 \land \phi_3 \land \phi_4$ is satisfiable.

  From trial and error, we see that it is satisfiable.  For example, set
  $p_3$ and $q_3$ to true, and all other variables to false.  That means
  Bertram can satisfy these requirements by taking 55 LEC 3 and 110 LEC 3,
  and no other classes.
\end{example}
