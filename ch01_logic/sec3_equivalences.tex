% -*- mode: latex; TeX-master: "ch1_logic_main.tex" -*-
\section{Propositional Equivalences}
\subsection{Truth assignments}
For the rest of today, we will step away from specific propositions
like ``if it is a holiday, then there is no class,'' and instead use
abstract variables, as in ``$p \limplies \lnot q$.''  We will focus on
how the truth value of compound statements like $p \limplies \lnot q$
depends on the truth values of $p$ and $q$.

\begin{table} \centering
  \label{tab:pimpnq}
  \begin{tabular}{| c | c | c | c |}
    \hline
    $p$ & $q$ & $p \limplies \lnot q$ & $q \limplies \lnot p$ \\ \hline
    T & T & F & F \\ \hline
    T & F & T & T \\ \hline
    F & T & T & T \\ \hline
    F & F & T & T\\ \hline
  \end{tabular}
  \caption{Truth table for $p \limplies \lnot q$ and $q \limplies \lnot p$}
\end{table}

\begin{defn}
  Given a list of propositional variables, a \emph{truth assignment}
  is a choice of ``true'' or ``false'' for each variable.
\end{defn}
In other words, a truth assignment picks out a single row in a truth
table.  For example, the assignment ``$p$ is false and $q$ is true,''
picks out the third row in Table~\ref{tab:pimpnq}.

Some propositions, like $p \lor \lnot p$, are true no matter what the
value of $p$ is.  Similarly, $p \land \lnot p$ is always false.  There
are special names for these situations:

\begin{defn}
  If a compound proposition is true for all possible truth
  assignments, it is called a \emph{tautology}.  If it is false for
  all truth assignments, it is a \emph{contradiction}.  If it is
  neither a tautology nor a contradiction, it is a \emph{contingency}.
\end{defn}

\subsection{Equivalence}
From Table~\ref{tab:pimpnq} we see that $p \limplies \lnot q$ and $q
\limplies \lnot p$ have the same truth table.  We would like to say
that these two propositions are ``essentially the same.''

\begin{defn}
  Two compound propositions $\alpha$ and $\beta$ are \emph{logically
    equivalent} if they have the same value for all truth
  assignments.  In this case write $\alpha \equiv \beta$.
\end{defn}

\begin{exr}
  Let $\alpha$ and $\beta$ be compound propositions.  Show $\alpha
  \equiv \beta$ if and only if $\alpha \liff \beta$ is a tautology.
\end{exr}

\begin{danger}
  The symbol $\equiv$ is \emph{not} a logical connective like $\land,
  \lor, \limplies$, and friends.  Similarly,
  $p \limplies \lnot q \equiv q \limplies \lnot p$ is not a compound
  proposition.  It is the ``meta-statement'' that
  $p \limplies \lnot q$ and $q \limplies \lnot p$ are equivalent
  propositions.  Some insight into why this must be: it would be
  nonsense to try to make a truth table for
  $p \limplies \lnot q \equiv q \limplies \lnot p$.
\end{danger}

\subsection{Satisfiablity}
\begin{defn}
  A compound proposition $\alpha$ is called \emph{satisfiable} if
  there exists a truth assignment for which $\alpha$ is true.  In
  other words, $\alpha$ is not a contradiction.
\end{defn}

Many concrete problems can be reduced to asking if some compound
proposition is satisfiable, and if it is, finding a satisfying
assignment.