% -*- mode: latex; TeX-master: "ch1_logic_main.tex"; fill-column: 75 -*-
\section{Predicates and Quantifiers}

The theory of Propositional Logic that we have developed so far, while useful, 
is not powerful enough to express many of the many of the statements we can 
make in mathematics and in the English language. For example, assuming the 
proposition ``Every university has a math department'' is true, we should 
be able to conclude that ``Berkeley has a math department'' is also true. 
However, we can't conclude this from propositional logic alone. 
To address this gap, we introduce \textbf{Predicate Logic}, which allows us 
to better reason about propositions which make assertions about sets of 
objects (like the above proposition made an assertion about the set of all 
universities).

\subsection{Predicates}

Consider the statement $x=y+3$. On its own, this statement isn't a proposition 
- we can't say whether it is true or false because we don't know the values 
of $x$ and $y$. However, if we also specify that $x=1$ and $y=2$, the 
statement becomes a proposition. This is an example of a \textbf{predicate}.

\begin{defn}
  A \textbf{predicate} is a statement which becomes a proposition when the values 
  of one or more variables are specified. These variables are called the 
  \textbf{subjects} of the predicate.
\end{defn}

\example Consider the following predicates:
\begin{enumerate}
  \item $x>3$
  \item $x=y+3$
  \item $x^2\geq 0$
\end{enumerate}

We use the capital letters $P,Q,R,\dots$ to specify predicates. For example, 
we could denote the predicate $x>3$ as $P(x)$ - the $x$ in parentheses is 
present to indicate that the predicate $P$ depends on the value of $x$. 
Given a specific value of $x$ - say $x=2$, the predicate becomes a proposition 
- for instance, $P(2)$ is the proposition $2>3$, which is false. 
$P(2)\equiv 2>3$ is also said to be the value of the 
\textbf{propositional function} $P$ at $x=2$.

\example Let $Q(x,y)$ be the predicate $x=y+3$. 
  Determine the truth values of $Q(1,2)$ and $Q(3,0)$.

\subsection{Quantifiers}

Given a predicate $P$, we saw above that by specifying a value of $x$, 
we get a proposition. There is another important way that we can obtain 
a proposition from a predicate, and that is via quantifiers. 
These quantifiers allow us to form propositions like: 
``$x^2\geq 0$ for any integer $x$'' or 
``There is a real number $x$ such that $x^2=2$''.

To introduce these quantifiers, we first have to discuss the 
\textbf{domain} of a predicate $P$. The domain the set of 
possible values for $x$ that we are considering. 
For instance, the domain of the predicate $x>3$ could be the set of real numbers.

\begin{defn}
The \textbf{universal quantification} of $P(x)$ is the statement: 
``$P(x)$ is true for all values of $x$ in the domain''. 
We can write this proposition more concisely as: 
$\forall x\:P(x)$ where $\forall$ is the \textbf{universal quantifier} and reads as `for all'.
\end{defn}

\example Let $P(x)$ be $x+1>x$. What is the truth value of 
  $\forall x\:P(x)$ where the domain consists of the real numbers?

\example Write the statement 
  ``Every computer in the computer lab is running Windows 10'' 
  as the universal quantification of a predicate. Specify the predicate and the domain.

\begin{defn}
  The statement $\forall x\:P(x)$ is false if and only if 
  we can find a value for $x$ in the domain such that $P(x)$ is false. 
  Such a value is called a \textbf{counterexample}.
\end{defn}

\example Let $Q(x)$ be the predicate ``$x<2$''. What is the truth value of $\forall x\:Q(x)$?

To show that a universal quantification $\forall x\:P(x)$ is false, 
we have to show that there exists a counterexample. 
This leads us to the existential quantifier:

\begin{defn}
  The \textbf{existential quantification} of $P(x)$ is the statement 
  ``There exists an element $x$ in the domain such that $P(x)$ is true''. 
  We can write this more concisely as $\exists x\:P(x)$ where $\exists$ reads as `there exists'.
\end{defn}

\example What is the truth value of $\exists x\:x>3$ where the domain is the set of real numbers?

There is a variant of the existential quantifier called the \textbf{uniqueness} 
quantifier which is denoted by $\exists !$. The statement 
$\exists !x\:P(x)$ expresses the proposition ``There is exactly $x$ such that $P(x)$ is true''.

\subsection{Working with Quantifiers}

Suppose you have a predicate $P$ with domain $\{1,2,3,4\}$ ($x$ can only take on the values $1,2,3,4$). 
Then the statement $\forall x\:P(x)$ amounts to saying $P(1),P(2),P(3)$, 
and $P(4)$ are all true - that is $P(1)\wedge P(2)\wedge P(3)\wedge P(4)$ is true. 
Thus $\forall x\:P(x)\equiv P(1)\wedge P(2)\wedge P(3)\wedge P(4)$. 
Similarly, the statement $\exists x\:P(x)$ amounts to saying that at least one of 
$P(1),P(2),P(3),P(4)$ is true, so $\exists x\:P(x)\equiv P(1)\vee P(2)\vee P(3)\vee P(4)$.

This line of reasoning extends to any finite domain. 
If the domain of the predicate $Q$ is a finite set, 
its elements can be listed as $x_1,x_2,\dots,x_n$, 
and then as above, $\forall x\:Q(x)\equiv Q(x_1)\wedge\dots\wedge Q(x_n)$ 
and $\exists x\:Q(x)\equiv Q(x_1)\vee\dots\vee Q(x_n)$.

$\newline$
Consider the predicate $x^2>0$ with domain the real numbers. 
If we want to express the proposition 
``$P(x)$ is true for every negative real number $x$'', 
we don't need to redefine the domain, we can simply write: 
``$\forall x<0\:(x^2>0)$''. 
In this way, we can restrict the part of the domain on which the quantifier applies.

\example What do the statements $\forall y\neq 0\:(y^3\neq 0)$ 
  and $\exists z>0\:(z^2=2)$ mean when the domain of each predicate is the real numbers?

\example (\textbf{Order of Operations}) 
  The quantifiers $\forall$ and $\exists$ take precedence 
  over all the logical operators from propositional calculus. 
  For instance the statement $\forall x\:P(x)\vee q$ means 
  $(\forall x\:P(x))\vee q$ rather than $\forall x\:(P(x)\vee q)$.

\subsection{Logical Equivalences involving Quantifiers}

When two compound propositions such as $\neg p\wedge \neg q$ and 
$\neg(p\vee q)$ are equivalent, they have the same truth value 
regardless of the truth assignment of $p$ and $q$. 
If $\phi$ and $\psi$ are propositions involving the predicates $P,Q,R,\dots$ and quantifiers, 
we say $\phi\equiv\psi$ if the two propositions have the same truth value 
independent of the predicates $P,Q,R,\dots$ (and their domains).

\example (\textbf{DeMorgan's Laws for Quantifiers}) The following logical equivalences hold:
\begin{enumerate}
  \item $\neg\forall x\:P(x)\equiv \exists x\:\neg P(x)$
  \item $\neg\exists x\:P(x)\equiv \forall x\:\neg P(x)$
\end{enumerate}
\begin{proof}
  We'll prove the first equivalence - $\neg\forall x\:P(x)\equiv \exists x\:\neg P(x)$ 
  - the second is left as an exercise. 
  The comment at the start of this section (3.4) means that 
  we can't assume anything about the predicate $P$.

  Let $\phi=\neg\forall x\:P(x)$ and $\psi=\exists x\:\neg P(x)$.
  To prove $\phi\equiv\psi$, it suffices to prove that
  $\phi\leftrightarrow\psi$ is a tautology. 
  Note $\phi\leftrightarrow\psi\equiv(\phi\rightarrow\psi)\wedge(\psi\rightarrow\phi)$,
  which is a tautology if and only if both $\phi\rightarrow\psi$ and $\psi\rightarrow\phi$
  are tautologies.

  Proof that $\phi\rightarrow\psi$ is a tautology: If $\phi$ is false,
  $\phi\rightarrow\psi$ is true, so we need only consider the case where $\phi$ is true.
  If $\phi$ is true, then $\forall x\:P(x)$ is false, so there exists a counterexample -
  there exists an $x$ such that $P(x)$ is false. But then there exists an $x$ such that
  $\neg P(x)$ is true, so $\psi=\exists x\:\neg P(x)$ is true. Thus $\phi\rightarrow\psi$.

  Proof that $\psi\rightarrow\phi$ is a tautology: If $\psi$ is false,
  $\psi\rightarrow\phi$, so we may assume $\psi$ is true.
  Then, there exists an $x$ such that $\neg P(x)$ is true, so there exists
  a counterexample to the proposition $\forall x\:P(x)$. Thus $\forall x\:P(x)$ is false,
  so $\phi=\neg\forall x\:P(x)$ is true. Thus $\psi\rightarrow\phi$.
\end{proof}

\example Show that $\forall x\:(P(x)\wedge Q(x))\equiv (\forall x\:P(x))\wedge(\forall x\:Q(x))$
  assuming $P$ and $Q$ have the same domain.
  
\example Negate the statements $\forall x\:(x^2>x)$ and $\exists x\:(x^2=2)$.
