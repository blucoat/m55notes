\documentclass[12t]{scrartcl}
\usepackage[letterpaper, portrait, margin=1in]{geometry}
\usepackage{fancyhdr}
\usepackage{amsmath,amsthm,amssymb}
\usepackage{enumerate}
\usepackage{hyperref}
\usepackage{array}
\usepackage{graphicx}
\usepackage{enumitem}
\setlist[enumerate]{itemsep=0mm}

\title{Math 55 Lecture Notes - Section 1.1}
\author{\vspace{-8ex}}
\date{\vspace{-8ex}}

\theoremstyle{plain}
\newtheorem{theorem}{Theorem}[section]
\newtheorem{lemma}[theorem]{Lemma}
\newtheorem{claim}[theorem]{Claim}
\newtheorem{corollary}[theorem]{Corollary}
\newtheorem{proposition}[theorem]{Proposition}
\newtheorem{conjecture}[theorem]{Conjecture}
\newtheorem{statement}[theorem]{Statement}
\newtheorem*{theorem*}{Theorem}
\newtheorem{cexample}[theorem]{Counterexample}
\newtheorem*{claim*}{Claim}

\theoremstyle{definition}
\newtheorem{definition}[theorem]{Definition}
\newtheorem{addendum}[theorem]{Addendum}
\newtheorem{example}[theorem]{Example}
\newtheorem{exercise}[theorem]{Exercise}
\newtheorem{remark}[theorem]{Remark}
\newtheorem{notation}[theorem]{Notation}
\newtheorem{problem}[theorem]{Problem}

\begin{document}
\maketitle

\section{Propositions}

\definition A \textbf{proposition} is a statement of fact that is either true or false.

\example Consider the following examples of propositions:
\begin{enumerate}[nosep]
	\item Toronto is the capital of Canada.
	\item $1+1=2$
	\item $2+2=3$
\end{enumerate}

\example The following `sentences' are not examples of propositions:
\begin{enumerate}[nosep]
	\item What time is it?
	\item $(2+4)/3$
	\item $x+1=2$
\end{enumerate}

\vspace{.25cm}
Notice how sentence 1 is not a statement of fact - it is a question. Sentence two is a mathematical expression, but is not an assertion (it is not making any claims), so it doesn't make sense of it to be true or false. Sentence $3$ is not fully defined, but it would be a proposition if we specified the value of $x$.

\textbf{Propositional variables} are variables that represent propositions. Conventionally, we use the letters $p,q,r,s,\dots$. Just as we use the letter $x$ to represent a number, we use the letter $p$ to represent a proposition. For instance, we could say that $p$ is the proposition that $1+1=2$. The \textbf{truth value} of a proposition can be true (abbreviated T) or false (abbreviated F) depending on whether the proposition is true or false.

\section{Logical Operators}

We can form new propositions from those we already have (like those given in example 1.2) using logical operators. The first, and simplest logical operator is the `not' operator.

\definition Let $p$ be a proposition. The \textbf{negation} of $p$, denoted by $\neg p$ (also denoted by $\bar{p}$), is the statement: ``$p$ is false''. The proposition $\neg p$ is read as ``not $p$''.

\example Express the negations of the following propositions
\begin{enumerate}[nosep]
    \item ``There are at least $10,000$ undergrads at Berkeley''.
    \item $1+1=2$
\end{enumerate}
\vspace{0cm}
\textbf{Solution:}
\begin{enumerate}[nosep]
    \item ``There are not at least $10,000$ undergrads at Berkeley'', or more simply, ``There are less than $10,000$ undergrads at Berkeley''.
    \item $1+1\neq 2$
\end{enumerate}

\vspace{.25cm}
The \textbf{truth table} for the negation of a proposition $p$ is a table containing the truth value of $\neg p$ for each of the $2$ possible truth values of $p$:
\begin{tabular}{l|r}
$p$ & $\neg p$ \\\hline
T & F \\
F & T
\end{tabular}

\definition Let $p$ and $q$ be propositions. The $\textbf{conjunction}$, or $\textbf{and}$ of $p$ and $q$, denoted $p\wedge q$, is the proposition ``$p$ and $q$''. $p\wedge q$ is true when both $p$ and $q$ are true, and false otherwise.

Consider the truth table of $p\wedge q$. Notice how it has $4$ rows, as to properly express the proposition $p\wedge q$ we have to consider all the possible combinations of truth values for $p$ and $q$.
\begin{tabular}{lc|r}
$p$ & $q$ & $p\wedge q$ \\\hline
T & T & T \\
T & F & F \\
F & T & F \\
F & F & F
\end{tabular}

\definition Let $p$ and $q$ be propositions. The $\textbf{disjunction}$, or $\textbf{or}$ of $p$ and $q$, denoted $p\vee q$ is the proposition ``$p$ or $q$''. $p\vee q$ is false when both $p$ and $q$ are false and true otherwise.

Consider the truth table of $p\vee q$:
\begin{tabular}{lc|r}
$p$ & $q$ & $p\vee q$ \\\hline
T & T & T \\
T & F & T \\
F & T & T \\
F & F & F
\end{tabular}

\example Express the proposition $p$:``It rained on Monday and Tuesday'' as a conjuction of 2 propositions.

\vspace{0cm}
\textbf{Solution:} If $q$ is the proposition ``It rained on Monday'' and $r$ is the proposition ``It rained on Tuesday'', then we can express $p$ as $q\wedge r$.

\exercise Give the truth tables for the following \textbf{compound propositions} (a compound proposition is a proposition built from smaller propositions using logical operators):
\begin{enumerate}[nosep]
    \item $\neg p\vee \neg q$
    \item $p \vee (q \wedge r)$
\end{enumerate}

Often, we use the word \textbf{or} in English exclusively - to mean ``one or the other, but not both''. This is not how the logical \textbf{or} is used - the logical \textbf{or} is always inclusive.

\definition Let $p$ and $q$ be propositions The \textbf{exclusive or} (XOR) of $p$ and $q$ denoted by $p\oplus q$ is the proposition that is true when exactly one of $p$ and $q$ is true and false otherwise.

The truth table of $p\oplus q$ is as follows:
\begin{tabular}{lc|r}
$p$ & $q$ & $p\oplus q$ \\\hline
T & T & F \\
T & F & T \\
F & T & T \\
F & F & F
\end{tabular}

\exercise Express $p\oplus q$ using the operators $\neg,\wedge,\vee$.

\section{Conditional Statements}

\definition Let $p$ and $q$ be propositions. The \textbf{conditional statement} is the proposition $p\rightarrow q$, read ``if $p$ then $q$'' or ``$p$ implies $q$''. $p\rightarrow q$ is false when $p$ is true and $q$ is false, and true otherwise.

Here is the truth table for $p\rightarrow q$:
\begin{tabular}{lc|r}
$p$ & $q$ & $p\rightarrow q$ \\\hline
T & T & T \\
T & F & F \\
F & T & T \\
F & F & T
\end{tabular}

\example The following are examples of conditional statements:
\begin{enumerate}[nosep]
    \item If $x=1$, then $x+1=2$.
    \item If it rains tomorrow, I will stay indoors.
    \item If $1+1=3$, then $1+1=2$.
    \item If $1+1=2$, then $1+1=3$.
\end{enumerate}

\vspace{.25cm}
From the conditional statement $p\rightarrow q$, we get three more propositions:
\begin{itemize}[nosep]
    \item The \textbf{converse} is the proposition $q\rightarrow p$.
    \item The \textbf{contraposative} is the proposition $\neg q\rightarrow\neg p$.
    \item The \textbf{inverse} is the proposition $\neg p\rightarrow \neg q$.
\end{itemize}

By observing the truth tables for $p\rightarrow q$, the converse, the contrapositive, and the inverse, we can see that $p\rightarrow q$ always has the same truth value as the contrapositive (this means $p\rightarrow q$ is \textbf{equivalent} to the contrapositive):\\
\begin{tabular}{lc|c|c|c|r}
$p$ & $q$ & $p\rightarrow q$ & $q\rightarrow p$ & $\neg q\rightarrow\neg p$ & $\neg p\rightarrow\neg q$ \\\hline
T & T & T & T & T & T \\
T & F & F & T & F & T \\
F & T & T & F & T & F \\
F & F & T & T & T & T
\end{tabular}

\example Find the converse, contrapositive, and inverse of the conditional statement ``If it rains tomorrow, I will stay indoors''.

\vspace{0cm}
\textbf{Solution:}
\begin{itemize}[nosep]
    \item \textbf{converse}: ``If I stay indoors tomorrow, it will rain''
    \item \textbf{contrapositive}: ``If I don't stay indoors tomorrow, it won't rain''
    \item \textbf{inverse}: ``If it doesn't rain tomorrow, I won't stay indoors''
\end{itemize}

The fact that the contrapositive is equivalent to the conditional statement will be useful later on in the course. If we are having trouble proving the proposition $p\rightarrow q$ directly, we might try to prove the contrapositive instead.

\definition Let $p$ and $q$ be propositions. The \textbf{biconditional statement} $p\leftrightarrow q$ is the proposition ``$p$ if and only if $q$''. It is true when $p$ and $q$ have the same truth values, and false otherwise.

Notice $p\leftrightarrow q$ is true when both $p\rightarrow q$ and $q\rightarrow p$ are true and values otherwise. That is, $p\leftrightarrow q$ is equivalent to $(p\rightarrow q)\wedge(q\rightarrow p)$.

\exercise Construct the truth table of the compound proposition $(p\vee\neg q)\rightarrow (p\wedge q)$. It may be useful to fill in extra columns for the simpler propositions $p\vee\neg q$ and $p\wedge q$ first.

\section{Additional Remarks}
The order of operations of logical operators is the following: $\neg,\wedge,\vee,\rightarrow,\leftrightarrow$. Therefore, $p\wedge q\vee r$ is $(p\wedge q)\vee r$ and $\neg p\wedge q$ is $(\neg p)\wedge q$.

\vspace{.25cm}
Often when working with logic we are working in a digital setting, and so it is helpful to encode truth values as bits. In this encoding, $1$ represents true and $0$ represents false.

\end{document}

